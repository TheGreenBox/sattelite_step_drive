\newpage
\subsection{ Основные принятые обозначения }

\begin{table}[ht!]
    \begin{tabular}{rll}
    $\theta$            & рад & текущее абсолютное положение ротора \\

    $I_{\text{ст}}$     & A & установившееся значение тока статора \\

    $L_{p}(\theta)$     & Гн & индуктивность ротора \\

    $L_{\text{ст}}(\theta)$     & Гн & индуктивность статора \\

    $L_{\text{р-ст}}(\theta)$   & Гн & взаимоиндукция ротора и статора \\

    $t_{1}$             & с & время подачи напряжения внутри импульса \\

    $T_\text{шим}$      & с & период ШИМ \\

    $\zeta$             & - & коэффициент заполнения ШИМ \\

    $T$                 & с & постоянная времения обмотки одной фазы \\

    $t$                 & с & текущее время \\

    $N_{r}$             & - & число пар полюсов ротора, безразмерная величина \\

    $i_\text{ред}$      & - & передаточное отношениие редуктора \\

    $p_{\text{шд}}$     & - & число фаз двигателя \\

    $J_{oy}$            & $\text{кг} \cdot \text{м}^{2}$ & момент инерции объекта управления \\

    $J$                 & $\text{кг} \cdot \text{м}^{2}$ & момент инерции объекта управления относительно оси привода \\

    $t_\text{п}$        & с & время перенацеливания камеры \\

    $\omega_{\beta.p}$  & $ \text{рад} \cdot \text{с}^{-1} $ & рабочая угловая
                            частота эквивалентного гармонического сигнала \\

    $\omega_{p}$        & $ \text{рад} \cdot \text{с}^{-1} $ & рабочая угловая
                            частота вращения ротора \\

    $\dot{\beta}_{max}$  & $\text{рад} \cdot \text{с}^{-1}$ & максимальная угловая скорость вращения камеры,
                            соответствующая параметрам эквивалентной синусоиды \\

    $\ddot{\beta}_{max}$ & $ \text{рад} \cdot \text{с}^{-2} $ & максимальное ускорение
                            вращения камеры, соответствующее параметрам эквивалентной синусоиды \\

    $A_{\text{экв}}$     & рад & амплитуда эквивалентного гармонического сигнала \\

    $T_{y.min}$          & с & минимальная длина импульса управления \\

    $f_{RC.\text{проп}}$ & Гц & ширина полосы пропускания RC-фильтра датчика тока \\

    $\psi_{M}$           & Вб & потокосцепление ротора шагового двигателя \\

    $K_{M}$              & $\text{А} \cdot \text{рад} \cdot \text{Н}^{-1} \cdot \text{м}^{-1}$ & моментный коэффициент \\

    $D_{\text{тр.сух}}$  & $\text{Н} \cdot \text{м}$ & момент сухого трения, действующего в системе \\

    $D_{\text{тр.вязк}}$  & $\text{Н} \cdot \text{м} \cdot \text{c} \cdot \text{рад}_{-1}$ & коэффициент вязкого трения, действующего в системе \\

    $K_{d1}$             & $\text{Н} \cdot \text{м}$ & амплитуда первой гармоники резонанса момента \\

    $K_{d2}$             & $\text{Н} \cdot \text{м}$ & амплитуда второй гармоники резонанса момента \\

    $K_{d4}$             & $\text{Н} \cdot \text{м}$ & амплитуда четвертой гармоники резонанса момента \\

    $\phi_{1}$           & рад & фаза первой гармоники резонанса момента \\

    $\phi_{2}$           & рад & фаза второй гармоники резонанса момента \\

    $\phi_{4}$           & рад & фаза четвертой гармоники резонанса момента \\

    \end{tabular}
    \caption{ Основные принятые обозначения }
\end{table}
