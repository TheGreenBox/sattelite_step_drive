\newpage
\subsection{ Основные принятые обозначения }

\begin{table}[ht!]
    \begin{tabular}{rll}
    $\theta$            & рад & текущее абсолютное положение ротора \\

    $I_{s}$             & A & установившееся значение тока статора \\

    $L_{r}(\theta)$     & Гн & индуктивность статора \\
    
    $L_{s}(\theta)$     & Гн & индуктивность статора \\
    
    $L_{sr}(\theta)$    & Гн & взаимоиндукция ротора и статора \\
    
    $t_{1}$             & с & время подачи напряжения внутри импульса \\
    
    $T_\text{ШИМ}$      & с & период ШИМ \\
    
    $\zeta$             & - & коэффициент заполнения ШИМ \\
    
    $T$                 & с & постоянная времения обмотки одной фазы \\
    
    $t$                 & с & текущее время \\
    
    $N_{r}$             & - & число пар полюсов ротора, безразмерная величина \\
    
    $i_\text{ред}$      & - & передаточное отношениие редуктора \\
    
    $p_{sm}$            & - & число фаз двигателя \\
    
    $J_{\text{ОУ}}$     & $\text{кг} \cdot \text{м}^{2}$ & момент инерции объекта управления \\
    
    $J$                 & $\text{кг} \cdot \text{м}^{2}$ & момент инерции объекта управления относительно оси привода \\
    
    $t_\text{п}$        & с & время перенацеливания камеры \\
    
    $\omega_{\beta.p}$  & $ \text{рад} \cdot \text{с}^{-1} $ & рабочая угловая
                            частота эквивалентного гармонического сигнала \\
    
    $\dot{\beta}_{max}$ & $\text{рад} \cdot \text{с}^{-1}$ & максимальная угловая
                            скорость вращения камеры, соответствующая параметрам эквивалентной синусоиды \\
    
    $\ddot{\beta}_{max}$ & $ \text{рад} \cdot \text{с}^{-2} $ & максимальное ускорение
                            вращения камеры, соответствующее параметрам эквивалентной синусоиды \\
    
    $A_{\text{экв}}$     & рад & амплитуда эквивалентного гармонического сигнала \\
    
    $T_{y.min}$          & с & минимальная длина импульса управления \\
    
    $f_{RC.\text{проп}}$ & Гц & ширина полосы пропускания RC-фильтра датчика тока \\
    
    $\psi_{M}$           & Вб & потокосцепление ротора шагового двигателя \\
    $K_{T}$              & $\text{А} \cdot \text{рад} \cdot \text{Н}^{-1} \cdot \text{м}^{-1}$ & моментный коэффициент \\

    \end{tabular}
    \caption{ Основные принятые обозначения }
\end{table}

