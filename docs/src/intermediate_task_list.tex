\newpage
\subsection{Промежуточные задачи}

\begin{enumerate}
    \item{Управление с обратной связью}
    \begin{enumerate}
        \item Определение коэффициета заполнения ШИМ внутри импульса управления
        \item Определение угла коммутации для текущей скорости в зависимости от того
            \begin{itemize}
                \item Разгон, ускорение больше нуля
                \item Торможение, ускорение меньше нуля
                \item Поддержание скорости постоянной
            \end{itemize}
        \item Алгоритм переключения фаз при работе с заданным углом коммутации
        \item Алгоритм для реализации обратной связи по току
    \end{enumerate}

    \item{Управление без обратной связи}
        \begin{enumerate}
            \item Определение коэффициета заполнения ШИМ внутри импульса управления
            \item Определение предельной скорости, ниже которой двигатель гарантировано не выйд из
                синхронизма при постоянной динамической нагрузке
        \end{enumerate}

    \item{Обратная связь по току}
        \begin{enumerate}
            \item Требования к реализации
            \item Анализ ограничений
            \item Исследование встроеного в микроконтролер АЦП
            \item Анализ путей решения зашумленности канала АЦП
        \end{enumerate}

    \item{Паразитные резонансные явления вшаговом двигателе}
        \begin{enumerate}
            \item Анализ возможных причин
            \item Выявление особо критичных параметров на качетсво системы управления
            \item Методы борьбы
        \end{enumerate}

    \item{Математическое моделирование}
        \begin{enumerate}
            \item Использование готовой математической модели из пакета MatLab Simulink
            \item Математическая модель в MatLab Simulink на основе уравнений электрических
                    процессов в фазах статора
            \item Проверка работоспособности и эффективности алгоритмов управления с заданным углом
                    коммутации на математических моделях
            \item Моделирование явлениий среднечастотного резонанса
            \item Методика валидации моделей по результатам натурного моделирования
        \end{enumerate}
\end{enumerate}

\newpage
