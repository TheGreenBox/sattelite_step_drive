\newpage
\subsection{Цели работы}

\begin{enumerate}
    \item{Разработка алгоритма управления ШД с обратной связью}
    \begin{enumerate}
        \item Определение коэффициета заполнения ШИМ внутри импульса управления
        \item Определение угла коммутации для текущей скорости в режимах:
            \begin{itemize}
                \item Разгон, ускорение больше нуля
                \item Торможение, ускорение меньше нуля
                \item Поддержание постоянной скорости, ускорение стремится к нулю
            \end{itemize}
        \item Алгоритм переключения фаз при работе с заданным углом коммутации
        \item Алгоритм для реализации обратной связи по току
    \end{enumerate}

    \item{Разработка алгоритма управления ШД без обратной связи}
        \begin{enumerate}
            \item Определение коэффициета заполнения ШИМ внутри импульса управления
            \item Определение предельной скорости, ниже которой двигатель гарантировано не выйдет из
                синхронизма при постоянной динамической нагрузке
        \end{enumerate}

    \item{Изучение вопроса об использовании ОС по току}
        \begin{enumerate}
            \item Требования к реализации
            \item Анализ ограничений
            \item Исследование встроеного в микроконтролер АЦП
            \item Анализ путей решения зашумленности канала АЦП
        \end{enumerate}

    \item{Выявление и изучение паразитных резонансных явлений в шаговом двигателе}
        \begin{enumerate}
            \item Анализ возможных причин
            \item Выявление особо критичных параметров для качетсва системы управления
            \item Методы борьбы, применимость для данной модели
        \end{enumerate}

    \item{Разработка и валидация математических моделей выбранного ШД}
        \begin{enumerate}
            \item Изучение готовой математической модели из пакета MATLAB Simulink
            \item Разработка своей математическая модель в MATLAB Simulink на основе
                    уравнений электрических процессов в фазах статора
            \item Проверка работоспособности и эффективности алгоритмов управления с заданным углом
                    коммутации на математических моделях
            \item Моделирование явлениий среднечастотного резонанса
            \item Разработка методики валидации моделей по результатам натурного моделирования
        \end{enumerate}
\end{enumerate}

\newpage
\subsection{Состав графической части}

\begin{enumerate}
    \item \textbf{Постановка задачи} - 1 лист
    \item \textbf{Функциональная схема блока управления} - 1 лист

    \item \textbf{Исследовательская часть}
    \begin{enumerate}
        \item Математическая модель шагового двигателя - 1 лист
        \item Блок-схемы алгоритма управления без обратной связи - 1 лист
        \item Блок-схемы алгоритма управления с использованием обратной связи по току - 1 лист
        \item Математическое моделирование в различных режимах работы - 1 лист
        \item Схема проведения и результаты натурного моделирования в различных режимах работы на моделирующем стенде - 2 листа
    \end{enumerate}

    \item \textbf{Конструкторская часть}
    \begin{enumerate}
        \item Плата управления
        \begin{enumerate}
            \item Принципиальная электрическая схема платы управления - 1 лист
            \item Разводка печатной платы блока управления - 1 лист
        \end{enumerate}
        \item Моделирующий стенд
        \begin{enumerate}
            \item Сборочный чертёж моделирующего стенда - 1 лист
            \item Деталировка стенда - 2 листа
        \end{enumerate}
    \end{enumerate}

    \item \textbf{Технологическая часть}
    \begin{enumerate}
        \item Технология производства платы управления - 2 листа
        \item Технология производства моделирующего стенда - 2 листа
    \end{enumerate}

    \item \textbf{Экономическая часть} - 2 листа
    \item \textbf{Экологическая часть} - 2 листа
    \item \textbf{Заключение} - 1 лист
\end{enumerate}

\subsection{Календарный план работы}

\begin{table}[ht!]
    \begin{tabular}{|c|l|l|}
    \hline
    № & Наименование этапа дипломного проектирования                            & Срок      \\ \hline
    1 & Теоретические исследования                                              & 25.02.11  \\ \hline
    2 & Выполнение конструкторской части                                        & 15.03.11  \\ \hline
    3 & Техническая подготовка экспериментов и отладка программного обеспечения & 25.04.11  \\ \hline
    4 & Проведение экспериментов и анализ результатов                           & 12.05.11  \\ \hline
    5 & Выполнение технологической части                                        & 19.05.11  \\ \hline
    6 & Выполнение организационно-экономической части                           & 23.05.11  \\ \hline
    7 & Выполнение части промышленной экологии и безопасности                   & 26.05.11  \\ \hline
    8 & Оформление работы в соответствие с требованиями                         & 01.06.11  \\ \hline
    \end{tabular}
\end{table}
