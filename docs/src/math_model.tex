\part{ Основные определения }
Максимальная частота приемистотости - максимальная управляющая частота, при которой не нагруженный
двигатель может останавливаться и запускаться без пропуска шагов.

Максимальная выходная чатота вращения - максимальная шаговая частота вращения, при которой
ненагруженный двигатель может двигаться без пропуска шагов.

Максимальный пусковой момент - максимальный момент сопротивления нагрузки, с которой двигатель может
запускаться и сохранять синхронность при частоте импульсов в 10 Гц.

\part{ Основные формулы }

Если не учитывать насыщение магнитной системы то справедливо равенство [ Л 1 : c 82 ]:

\begin{equation}
\label{step_motor_torque_common}
    M(\theta)
    = \frac{dW_r}{d\theta_m} 
    = p \frac{dW_s}{d\theta}
    = \frac{1}{2} p I_s \frac{dL_s}{d\theta}
        + \frac{1}{2} p I_r \frac{dL_r}{d\theta}
        + \frac{1}{2} p I_{s} I_r \frac{dL_{sr}}{d\theta}
\end{equation}

$I_{s}, I_{r}$ - установившееся значение токов статора и ротора соответсвенно

$L_{s}(\theta)$ - индуктивность статора

$L_{r}(\theta)$ - индуктивность ротора

$L_{sr}(\theta)$ - взаимоиндукция ротора и статора

Частота собственных упругих колебаний ротора \cite[гл 3-1]{Chilikin} при малых отклонениях от положения устойчивого
положения позволяет получить сторогую оценку механической подвижности системы и является ее мерой.

\begin{equation}
\label{step_motor_torque_without_load_and_with_unstable_rotor}
    M(\theta)
    = - M_{m} \sin{\theta}
    = - M_{m} \sin{p\theta_{M}}
\end{equation}

\begin{equation}
\label{step_motor_dynamic_move_equation}   
    J \ddot{ \theta_{M} } + M_{m} \sin{p \theta_{M}} = 0
\end{equation}

При малых отклонениях от положения равновесия таких что:
\begin{equation}
    \sin{\theta} \thickapprox 0
\end{equation}

\begin{equation}
    \label{rotor_like_harmonical_oscilator_equation}
    \ddot{\theta} + \frac{p M_{m}}{J} \theta = 0 
\end{equation}

\begin{equation}
    \label{friquent_for_rotor_self_oscilating}
    \omega = \sqrt{\frac{p M_{m}}{J}}
\end{equation}

\newpage
\endinput

