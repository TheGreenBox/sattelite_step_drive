\newpage
\part{Постановка задачи}
\section{Назначение и общий вид манипулятора}
Целью данной курсовой работы ставится проектирование одного из двух 
приводов манипулятора, устанавливаемых на малых спутниках съемки Земли (рис. 1).
Манипулятор представляет собой двухзвенный электроприводной механизм,
предназначенный для скоростного наведения оптических камер. Задачей манипулятора 
является организация перенацеливания камеры в заданное положение. Программные 
движения манипулятора осуществляются с помощью двух электроприводных блоков,
обеспечивающих поворот камеры в одной плоскости (в плоскости перпендикулярной
вектору движения космического аппарата по орбите).
В данный момент манипуляторы подобного рода на спутниках широко востребованы 
в областях картографирования, планировки территорий, образовательных, 
разведывательных и военных целях, метеорологии и т.п. 
К подобным манипуляторам предъявляют высокие требования по точности,
надежности, массе, габаритам.

\section{Описание объекта управления}
Проектируемый привод установлен на малом спутнике Земли и представляет собой 
исполнительный элемент системы дистанционного зондирования Земли (далее - ДЗЗ).
Объектом управления является специальная камера, предназначенная для создания 
снимком поверхности Земли высокой четкости, оборудована двумя звёздными 
датчиками для ориентации в пространстве и находящаяся внутри корпуса
экранно-вакуумной теплоизоляции (далее - ЭВТИ).

Общая масса камеры и ее навесных элементов: 60.5 кг. С учетом запаса, выберем
массу объекта управления манипулятора: 80 кг.
Расстояние от оси привода до центра масс камеры: 400 мм (с учетом запаса 
на более габаритную камеру и запаса на ускорение камеры);
Собственный момент инерции объекта управления вокруг оси, параллельной оси привода: 
$$
    J_{pl} = 2.84 kg \cdot m^2
$$

Момент инерции относительно оси привода:
$$
    15.7 kg \cdot m^2
$$

%\section{Требуемые параметры перенацеливания приводных блоков}

\section{Требуемые параметры привода}

\begin{tabular}{|l|c|l|}
\hline
Параметр                                    & Обозначение      & Значение \\
\hline
Напряжение питания                          & $U_1$            & 24В \\
Шаг единичного углового перемещения нагрузки& $q_d$            & $ \le 1' $ \\
Ресурс                                      &                  & 30 000 ч. \\
Диапазон углового перенацеливания нагрузки  & $q_{max}$        & $[-135^\circ, 135^\circ] $\\
Максимальная угловая скорость нагрузки      & $\dot{q}_{max}$  & $1.73$ рад / c \\
Максимальное ускорение нагрузки             & $\ddot{q}_{max}$ & $0.872$ рад /$c^2$ \\
Момент инерции нагрузки                     & $J_{pl}$         & $3 kg \cdot m^2 $\\
\hline
\end{tabular}

\endinput

