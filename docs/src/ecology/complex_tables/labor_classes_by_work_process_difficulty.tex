\renewcommand{\tabularxcolumn}[1]{m{#1}}
\newcolumntype{C}{>{\centering\arraybackslash}X}
\newcolumntype{L}{>{\raggedright\arraybackslash}X}

\begin{table}[ht]
    \centering
    % \begin{tabular}{|p{3.5cm}|c|c|c|c|}
    \begin{tabularx}{\textwidth}{|L|C|C|C|C|}
        \hline
        \multirow{4}{*}{\parbox{\linewidth}{\centering Показатели тяжести трудового процесса}}
        & \multicolumn{4}{c|}{Классы условий труда}                                 \\ \cline{2-5}

        & \multirow{3}{*}{\parbox{\linewidth}{Оптимальный (легкая физ. нагрузка)}}
        & \multirow{3}{*}{\parbox{\linewidth}{Допустимый (средняя физ. нагрузка)}}
        & \multicolumn{2}{c|}{Вредный (тяжелый труд)}                               \\
        & & & \multicolumn{2}{c|}{}                                                 \\ \cline{4-5}
        & &                                          & 1-й степени & 2-й степени    \\ \cline{2-5}

                                                            & 1 & 2 & 3.1   & 3.2   \\ \hline

        Физическая динамическая нагрузка                    & * &   &       &       \\ \hline
        Масса поднимаемого и перемещаемого груза вручную    & * &   &       &       \\ \hline
        Стереотипные рабочие движения                       &   & * &       &       \\ \hline
        Статическая нагрузка                                & * &   &       &       \\ \hline
        Рабочая поза                                        &   & * &       &       \\ \hline
        Наклоны корпуса                                     & * &   &       &       \\ \hline
        Перемещения в пространстве                          & * &   &       &       \\ \hline
    \end{tabularx}
    \caption{Классы условий труда по показателям тяжести трудового процесса}
    \label{labor_classes_by_work_process_difficulty_tbl}
\end{table}
