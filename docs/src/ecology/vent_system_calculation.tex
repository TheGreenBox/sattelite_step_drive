\newpage

\subsection{Расчёт системы вентиляции и местного вытяжного устройств}

Пары припоя и флюса, образующиеся при пайке, оказывают вредное воздействие на
организм человека и на окружающую среду. В состав используемого припоя ПОС-61
входит свинец, по физиологическому воздействию относящийся к группе соматических
ядов, вызывающих нарушения деятельности организма, его отдельных органов и систем.
По классификации из ГОСТ 12.1.005-88 \cite{ecology_gost_005_88}, свинец относится
к классу чрезвычайно опасных веществ с предельно допустимой концентрацией (ПДК)
в воздухе менее 0,1 $\text{мг/м}^3$ (ПДК свинца - 0,003 $\text{мг/м}^3$). Свинец
и его соединения вызывают изменения в сердечно-сосудистой и нервной системах,
снижают иммунобиологическую активность человека, а также нарушения ферментативных
реакций, витаминного обмена. Наиболее частыми формами отравления свинцом являются
малокровие, плеврит, свинцовые колики и гепатит.

Так как концентрация вредных паров, выделяющихся в процессе пайки, значительно
превышает ПДК (примерно в $2 - 4$ раза), необходимо применять местную вытяжную вентиляцию.
У каждого рабочего места, где выполняется связанная с пайкой печатной платы часть
технологического процесса, устанавливается всасывающая панель.

Всасывающая панель – это приспособление, применяемое в качестве местного отсоса
при таких ручных операциях, как электросварка, газовая сварка, резка металла,
пайка и т.п. При этом «зеркало» всасывания расположено наклонно к рабочему месту,
что не позволяет попадать вредным веществам в зону дыхания рабочего.

Объём воздуха, удаляемый панелью, находим по формуле \ref{air_volume_to_vent}:

\begin{equation}
\label{air_volume_to_vent}
    L = F_\textit{п} \cdot V_\textit{п}
\end{equation}

где $F_\textit{п}$ - площадь рабочего проёма, через который засасывается воздух, $\textit{м}^2$

$V_\textit{п}$ - скорость воздуха в рабочем проёме панели, $\frac{\text{м}}{\text{с}}$

Скорость воздуха во всасывающем факеле панели при удалении вредных испарений с
ПДК $q < 1$ $\text{мг/м}^3$ \cite[табл. 1.1]{local_vent_spot_calc_method} принимается $2 - 3,5$ м/c, примем $V_\textit{п}$ = 3 м/с.

Необходимый воздухообмен найдём по формуле:
