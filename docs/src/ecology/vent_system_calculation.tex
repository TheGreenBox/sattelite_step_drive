\newpage

\subsection{Расчёт системы вентиляции и местного вытяжного устройств}

Процесс производства печатной платы сопровождается большим количеством паечным
операций.
Пары припоя и флюса, образующиеся при пайке, оказывают вредное воздействие на
организм человека и на окружающую среду. В состав используемого припоя ПОС-61
входит свинец, по физиологическому воздействию относящийся к группе соматических
ядов, вызывающих нарушения деятельности организма, его отдельных органов и систем.
По классификации из ГОСТ 12.1.005-88 \cite{ecology_gost_005_88}, свинец относится
к классу чрезвычайно опасных веществ с предельно допустимой концентрацией (ПДК)
в воздухе $q_\text{пдк}$ = 0,003 $\text{мг/м}^3$.
Свинец и его соединения вызывают изменения в сердечно-сосудистой и нервной системах,
снижают иммунобиологическую активность человека, а также нарушения ферментативных
реакций, витаминного обмена. Наиболее частыми формами отравления свинцом являются
малокровие, плеврит, свинцовые колики и гепатит.

Так как концентрация вредных паров, выделяющихся в процессе пайки, значительно
превышает ПДК (примерно в $2 - 4$ раза), необходимо применять местную вытяжную вентиляцию.
У каждого рабочего места, где выполняется связанная с пайкой печатной платы часть
технологического процесса, устанавливается всасывающая панель.

Всасывающая панель – это приспособление, применяемое в качестве местного отсоса
при таких ручных операциях, как электросварка, газовая сварка, резка металла,
пайка и т.п. При этом «зеркало» всасывания расположено наклонно к рабочему месту,
что не позволяет попадать вредным веществам в зону дыхания рабочего.

\subsubsection{Расчёт общих параметров системы вентиляции}

Объём воздуха, удаляемый всасывающей панелью, находим по формуле
(\ref{air_volume_to_vent}):

\begin{equation}
\label{air_volume_to_vent}
    L = F_\text{п} \cdot V_\text{п}
\end{equation}

где $F_\text{п}$ - площадь рабочего проёма, через который засасывается воздух, $\text{м}^2$

$V_\text{п}$ - скорость воздуха в рабочем проёме панели, $\text{м/с}$

Скорость воздуха во всасывающем факеле панели при удалении вредных испарений с
ПДК $q < 1$ $\text{мг/м}^3$ \cite[табл. 1.1]{local_vent_spot_calc_method} принимается
$2 - 3,5 \text{ м/c}$, примем $V_\text{п} = 3 \text{ м/c}$.

Необходимый воздухообмен найдём по формуле (\ref{required_air_exchange}):

\begin{equation}
\label{required_air_exchange}
    L = \frac{G}{q_\text{выт} - q_\text{пр}}
\end{equation}

где G - интенсивность выделения вредных испарений, $\text{мг/час}$,
$q_\text{выт}$ – концентрация вредных испарений в удаляемом воздухе, $\text{мг/м}^3$
$q_\text{пр}$ – концентрация вредных испарений в приточном воздухе, $\text{мг/м}^3$

Согласно \cite[п. 2.15]{ecology_san_norm_245_71}:
\begin{equation}
\label{dangeroud_vapor_concentrations}
    \begin{array}{lcr}
        q_\text{пр}  & \leq & 0,3 \cdot q_\text{пдк} = 9 \cdot 10^{-4} \text{ мг/м}^3 \\
        q_\text{выт} & \leq &           q_\text{пдк} = 3 \cdot 10^{-3} \text{ мг/м}^3
    \end{array}
\end{equation}

Для расчётов примем выделение свинца при пайке $G = 18 \text{ мг/час}$,
тогда для максимальных значений концертраций вредных испарений из
(\ref{dangeroud_vapor_concentrations}), необходимый по формуле (\ref{required_air_exchange})
воздухообмен на рабочем месте:

$$
    L = 8570 \text{ мг/м}^3
$$

Тогда из формулы (\ref{air_volume_to_vent}) получим площадь рабочего проёма:

\begin{equation}
\label{working_window_area}
    F_\text{п} = \frac{L}{V_\text{п}} = 0,8 \text{ м}^2
\end{equation}

Найдем длины сторон А и B рабочего проёма исходя из того, что $F_\text{п}  = A \cdot B$ и приняв

\begin{equation}
\label{working_window_area_sides_ratio}
    \frac{A}{B} = \frac{3}{4}
\end{equation}

Тогда получим
$$
    A = 775 \text{ мм}
$$
$$
    B = 1033 \text{ мм}
$$

% TODO: вставить рисунок панели равномерного всасывания

Площадь проходного (живого) сечения таких панелей рекомендуется брать в 4,4
раза меньше габаритной площади, т.е.

\begin{equation}
\label{alive_section_are}
    f_\text{ж} = \frac{F_\text{п}}{4,4} = 0,18 \text{ м}^2
\end{equation}

Общий объём вытяжки будет равен

\begin{equation}
\label{overall_sucktion_volume}
    L_{\sum} = n \cdot f_\text{ж} \cdot V_\text{в}
\end{equation}

где $n$ - число рабочих мест,
$n$ - число рабочих мест,
$V_\text{в}$ - скорость воздуха, $\text{м/с}$

Приняв скорость воздуха в живом сечении равной $V_\text{в} = 3 \text{ м/с}$,
для $n = 6$-ти рабочих мест получим

$$
    L_{\sum} = 3,24 \text{ м}^3 / \text{с}
$$

Расчётная производительность вентилятора с учётом потерь и подсоса воздуха
в воздуховодах

\begin{equation}
\label{fan_productivity}
    L_s = 1,1 \cdot L_{\sum}
        \approx 3,56 \text{ м}^3 / \text{с}
        \approx 12,83 \cdot 10^3 \text{ м}^3 / \text{ч}
\end{equation}

\subsubsection{Расчёт отдельных участков системы вентиляции}

\begin{figure}[ht]
    \centering
    \includegraphics[width=\textwidth, keepaspectratio, clip=true, trim=0mm 35mm 0mm 40mm]
                    {./src/ecology/pictures/vent_system_arrangement}
    \caption{Расчётная схема системы вентиляции}
    \label{pic_vent_system_arrangement}
\end{figure}

Рассчитаем следующие параметры для каждого участка в соотвтетствии со схемой на
рисунке (\ref{pic_vent_system_arrangement}):

\subfile{src/ecology/vent_parameters}

\subsubsection{Выбор вентиляционного оборудования}

\paragraph{Вентилятор}
По полученным значениям общей производительности $L_s = 12,83 \cdot 10^3 \text{ м}^3 / \text{ч}$
и давлением $P_s = 772,56 \text{ Па}$ выберем центробежный
вентилятор ВЦ 14-46 из каталога компании «БПО» с частотой вращени
$n = 730 \text{ об/мин}$, производительностью $13,0 \cdot 10^3 \text{ м}^3 / \text{ч}$
и полным давлением 980 Па.

\paragraph{Двигатель}
Вентилятор укомплектован двигателем АИР132М8 мощность 5,5 кВт.

\paragraph{Фильтр}
Для проектируемой системы выбираем волокнистый ячейковый фильтр.

Максимальная концентрация пыли в рабочей зоне $z_\text{раб.зона} = 0,5 \text{ мг/м}^3$.
Содержание пыли в наружном воздухе непромышленного города $z_\text{город} = 0,6 \text{ мг/м}^3$.

Требуемя эффективность очистки приточного воздуха:

$$
    E_\text{тр} = \frac{z_\text{город} - z_\text{раб.зона}}{z_\text{город}} \cdot 100 \% = 17 \%
$$

Данная степень очистки соответствует классу фильтра тонкой очистки F5, предел
эффективности - 60\% по ГОСТ Р 51251-99 \cite{ecology_gost_51251_99}.
Подберем фильтр по данным \cite{air_ventilation_and_conditioning}[прилож. IV, табл. IV.1]:

\begin{table}[ht]
    \centering
    \begin{tabular}{l|l}
        \hline
        Тип фильтра                                         & волокнистый, замасляный ячейковый ФяУБ            \\
        Фильтрующий материал                                & стекловолокно ФСВУ                                \\
        Номинальная воздушная нагрузка на входное сечение   & $q = 7000 \text{ м}^3/(\text{ч} \cdot \text{м}^2)$\\
        Площадь ячейки                                      & $f_\text{я} = 0,22 \text{м}^2 $                   \\
        Начальное сопротивление                             & $P_\text{ф.н} = 40 \text{ Па}$                    \\
        Конечное сопротивление                              & $P_\text{ф.к} = 150 \text{ Па}$                   \\
        Удельная пылемкость                                 & $\text{П} = 570 \text{ г/м}^2$                    \\
        Способ регенерации                                  & замена фильтрующего материала.                    \\
        \hline
    \end{tabular}
    \caption{Характеристики фильтра}
\end{table}

Требуемая площадь фильтрации в соответствии с формулой (\ref{required_air_exchange}):
$$
    F_\text{ф.тр} = \frac{L}{q} = 1,22 \text{ м}^2
$$
где L - необходимый воздухообмен.

Необходимое количество ячеек:
$$
    n_\text{я} =    \frac{F_\text{ф.тр}}{f_\text{я}} = 5,56
                    \simeq 6
$$

Действительная степень очистки по номограмме
\cite{air_ventilation_and_conditioning}[прилож. IV, рис. IV.4] $\eta_\text{д} = 84 \%$, что
больше, чем $\eta_\text{тр} = 1 - E_\text{тр} = 83 \%$.

Количество пыли, осаждаемой на $1 \text{ м}^2$ площади фильтрации в течение 1 часа:
$$
    m_\text{уд} = \frac{L \cdot z_\text{город} \cdot \eta_\text{д}}{F_\text{ф.тр}}
            = 1,05 \text{ г/}(\text{м}^2 \cdot \text{ч})
$$

Периодичность замены фильтрующей поверхности:
$$
    \tau_\text{зам} = \frac{\text{П}}{m_\text{уд}} = 23 \text{ сут}
$$

Данная вытяжная система позволяет удалять из воздуха на рабочем месте вредные
вещества до содержания их в пределах допустимой концентрации. Правильная эксплуатация
системы предусматривает периодическое обследование состояния воздушной среды и
элементов вентиляционных установок, а также их правильное обслуживание, своевременную
очистку фильтров воздуховодов, проведение планового ремонта.
