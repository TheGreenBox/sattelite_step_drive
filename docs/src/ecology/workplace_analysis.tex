\newpage
\section{Промышленная экология и безопасность}

\subsection{Анализ условий труда на рабочем месте инженера-разработчика}

\subsubsection{Введение}

В рамках данного дипломного проекта проводится проектирование системы управления
манипулятора, расположенного на борту спутника, на основе шагового привода.
Основным рабочим местом является стол, оборудованный персональной ЭВМ
(далее - ПЭВМ) с визуально-дисплейным терминалом (далее - ВДТ) и макетом
манипулятора, который состоит из платы управления, экспериментального стенда с
имитацией нагрузочных моментов, внешнего источника питания и шагового двигателя.

% TODO: сделать и вставить схему рабочего места

Помимо специфических условий зрительной работы, напряженного
нервно-эмоционального характера труда, вынужденной рабочей позы, недостатка
подвижности и физической активности, работающие за ВДТ подвергаются воздействию
низкоэнергетического УФ и рентгеновского излучений, шума, электромагнитных и
электростатических полей, неудовлетворительного микроклимата, вентиляции и
освещения.

\subsubsection{Требования к помещениям для эксплуатации ПЭВМ и ВДТ}
