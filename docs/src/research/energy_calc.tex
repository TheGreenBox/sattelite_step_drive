\subsubsection{Энергетический расчёт привода}
\paragraph{Динамический момент на выходном валу привода}

Уравнение максимального динамического момента однозвенного механизма имеет вид

\begin{equation}
    \mu_\textit{д.max} = J_\textit{оу} \cdot \ddot{q}_{max} + \mu_\textit{сопр}
    \label{one_axis_drive_dynamical_torque}
\end{equation}

Подставив в формулу (\ref{one_axis_drive_dynamical_torque}) данные из табл.
\ref{drive_parameters_tbl}, а так же приняв коэффициент запаса момента на трение,
сопротивление изгибу кабельных линий, и других моментов сопротивления примем
равным $K_\textit{запас.сопр} = 1.05$, получим

$$
    \mu_\textit{д.max} = K_\textit{запас.сопр} (J_\textit{оу} \cdot \ddot{q}_{max}
                        + \mu_\textit{сопр}) = 21.3 \text{ Н$\cdot$м}
$$

\paragraph{Требуемый момент на выходном валу привода}

Проведём предварительный расчёт, предполагая, что будет использоваться редуктор
с передаточным числом $i_\textit{ред} = 100$. КПД таких редукторов как правило
составляет около 80\%. Примем коэффициент запаса на КПД такого редуктора равным
$K_\textit{запас.ред} = 1 / 0.8 = 1.25$

\subparagraph{Запас по рекомендациям Б.А. Ивоботенко}
В соответствии с рекомендациями \cite{IvobotenkoKazachenko}, момент сопротивления
нагрузки, приведенный к валу шагового двигателя, должен удволетворять условию

$$
    M_\textit{н}/i \leq 0.4 M_{max}
$$
где $M_{max}$ – максимальный момент шагового двигателя (как один из параметров).
Момент инерции нагрузки должен быть приблизительно равен моменту инерции ротора,
т.е.

$$
    \frac{J_\textit{н}}{i^2} \approx J_\textit{р}
$$

Исходя из первой рекомендации Б.А. Ивоботенко, коэффициент запаса для расчета
требуемого момента на валу электропривода должен быть равен

$$
    K_\textit{запас.н} = 1 / 0.4 = 2.5
$$

При выборе двигателя с таким коэффициентом запаса по моменту мы получим повышенную
мощность потребления, а также повышенные массогабаритные характеристики двигателя.
Поэтому целесообразно выбрать запас

$$
    K_\textit{запас.н} = 1.5
$$

Вторая рекомендация (о моменте инерции нагрузки) не понятна. Здесь, вероятно,
актуальна выдержка из книги Чиликина \cite{Chilikin}: ``Важно, что выявлена верхняя
граница разумного нагружения двигателя, т.к. при нагрузках превышающих определенную
величину, происходят потери не только в быстродействии, но и в передаваемой мощности''.

\subparagraph{Запас по рекомендациям Telco Motion}
По рекомендациям Telco Motion, необходимо учитывать, что в космическом исполнении
момент ШД будет на 30\% меньше, чем в нормальном исполнении.

Текст рекомендации:
\begin{otherlanguage}{english}
    \textit{For space application with a wide temperature range, the motor torque is reduced
    as we must allow extra mechanical clearance in the magnetic circuit to allow for
    the dimensional changes as the result of temperature variation. Typically this
    is about a 30\% torque reduction but the power does not change.}
\end{otherlanguage}

Исходя из данной рекомендации, коэффициент запаса для космического исполнения
двигателя должен быть равен

$$
    K_\textit{запас.косм} = 1 / 0.7 = 1.43
$$

Тогда требуемый момент на выходном валу привода

$$
    \mu_\textit{треб} =
                K_\textit{запас.ред} \cdot K_\textit{запас.косм}
                \cdot K_\textit{запас.н} \cdot \mu_\textit{д.max}
                = 57 \text{ Н$\cdot$м}
$$

\paragraph{Расчет момента шагового двигателя}

Момент шагового двигателя определяется как
\begin{equation}
    \mu_\textit{шд} = \mu_\textit{треб} / i_\text{ред}
    \label{stepper_engine_torque}
\end{equation}


При $i_\text{ред} = 100$ требуемый момент шагового двигателя
$$
    \mu_\textit{шд} = 0.57 \text{ Н$\cdot$м}
$$

Для космических аппаратов актуально снижение потребляемой мощности, поэтому
целесообразно снижение требуемого момента шагового двигателя
(следовательно и мощности приводного блока), что возможно реализовать при более
высоком передаточном числе редуктора. У компании Empire Magnetics есть подходящие
редукторы с передаточным числом 225.

При $i_\text{ред} = 225$ требуемый момент шагового двигателя
$$
    \mu_\textit{шд} = 0.25 \text{ Н$\cdot$м}
$$

Остановимся на редукторе с передаточным числом $i_\text{ред} = 225$.
