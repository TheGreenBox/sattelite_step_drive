\newpage
\part{Моделирование}

\section{Математическое моделирование}

\subsection{Моделирование в MATLAB}
\subsubsection{Модель ``Hybrid Stepper Motor'' из пакета ``SimPower Systems''}

Для моделирования процессов, протекающих в системе, воспользуемся пакетом MATLAB.
Перед построением модели всей системы, необходимо получить все параметры,
необходимые для использования встроенной модели гибридного шагового двигателя из пакета:

\textit{SimPower Systems -> Second Generation -> Motors and Generators -> Hybrid Stepper Motor.}

Известно всё необходимое, за исключением величины потосцепления, формируемого
постоянными магнитами ротора. Его можно определить по формуле 
из \cite{Matlab_help_stepper_motor}:

\begin{equation}
    \label{maximum_flux_linkage}
    \psi_{M} = \frac{30}{\pi p} E_{M} N
\end{equation}

$N$ - частота вращения ротора, об/мин

$$
    p = \frac{360}{2m \cdot step}
$$

$m = 2$ - число фаз шагового двигателя

$step = 1,8$ - шаг углового перемещения шагового двигателя, град

$E_{M}$ - амплитудное значение ЭДС самоиндукции разомнутой обмотки статора, В

Для использования (\ref{maximum_flux_linkage}) проведём следующий эксперимент:
с помощью дополнительного двигателя на стенде разгоним интересующий нас шаговый 
двигатель до частоты вращения $N = const$ и с помощью осцилографа определим $E_{M}$
в одной из его разомкнутых обмоток.

\subsubsection{Модель в MATLAB Simulink}


\newpage
\section{Натурное моделирование}

\endinput

