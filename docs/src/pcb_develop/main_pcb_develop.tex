\documentclass{article}

\usepackage[utf8]{inputenc}
% Выбор внутренней TEX−кодировки
\usepackage [T2A]{fontenc}
% Включение переносов для русского и английского языков
\usepackage[english,russian]{babel}

% Подключение гиперссылок
\usepackage{xcolor}
\usepackage[unicode]{hyperref}

% Настройка внешнего вида гиперссылок
\definecolor{LINKCOLOUR}{rgb}{0.1,0.0,0.9}
\hypersetup{colorlinks,breaklinks,urlcolor=LINKCOLOUR,linkcolor=LINKCOLOUR}

% Начинать первый параграф раздела следует с красной строки
\usepackage{indentfirst}

% Дополнительные математические пакеты
%\usepackage{weird,querr}
\usepackage{amssymb}
\usepackage{amsmath}

% Для корректного копирования из документа
\usepackage{cmap}

% Кто нибудь помнит зачем это тут? A: Неа...
% это многострочный текст, мы используем для многострочности в ячейках таблицы
\usepackage{multirow}

% стилевой пакет, открывающий доступ к большому числу типографских значков
\usepackage{textcomp}

% Меняем поля страницы
\usepackage{geometry}
\geometry{left=2cm}    % левое поле
\geometry{right=1.5cm} % правое поле
\geometry{top=1.5cm}     % верхнее поле
\geometry{bottom=1.5cm}  % нижнее поле

\usepackage{subfiles}

% работа с импортом изображений
\ifx\pdfoutput\undefined
\usepackage{graphicx}
\else
\usepackage[pdftex]{graphicx}
\fi

% секции и их структура
\usepackage[section]{placeins}
\usepackage{subcaption}

\begin{document}
\tableofcontents
\newpage
\section{Разработка печатной платы}
\begin{flushright}
    \itshape
    There is only the Emperor, and he is our shield and protector.\\
    \ldots
\end{flushright}
Технологичность по \textit{ГОСТ 14.205-83} рассматривается, как совокупность
свойств конструкции изделия, проявляемых в возможности оптимальных затрат труда
средств, материалов и времени при технологической подготовке производства,
изготовлении, эксплуатации и ремонте по сравнению с соответствующими
показателями однотипных устройств и изделий того же самого назначения, при
обеспечении установленных значений показателей качества в принятых условиях
изготовления, эксплуатации и ремонта.

\subfile{src/pcb_develop/pcb_technology.tex}
%\subfile{src/pcb_develop/pcb_design.tex}

\newpage
\section[Список использованной литературы]{}
\begin{thebibliography}{30}
    \subfile{src/pcb_develop/bibliography}
\end{thebibliography}

\end{document}
