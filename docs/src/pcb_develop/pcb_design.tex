\newpage
\subsection{Разработка платы управления}
%\subsubsection{Перечень требований}
\subsubsection{Реализация подсистем}
\paragraph{Вычислительное устройство платы управления}
Критерии выбора:
\begin{itemize}
    \item Быстродействие. Определяется частотой МП. Максимальная скорость
            вращения привода равна $99 \frac{\textdegree}{\text{с}}$. Для самого
            тяжелого с точки зрения процессорного времени алгоритма ---
            ``half step mode'' имеем шаг $0.9\textdegree$. Тогда частота
            следования импульсов равна $110 \text{Гц}$. В обработчике прерывания
            импульса от 100 до 10000 тактов работы процессора. С учетом работы
            остальной периферии, токового контура и некоторым коэффициентом
            запаса примем частоту работы не менее $32 \text{МГц}$
    \item Наличие модуля ШИМ
    \item Наличие АЦП (не менее 10-ти разрядов)
    \item Наличие прерываний периферии для организации прерываний по таймеру для
            тактирования импульсов шагового двигателя и реализации работы с
            импульсным датчиком угла
    \item Наличие последовательного интерфейса (UART) для взаимодействия с
            бортовым компьютером
    \item Наличие 32-разрядных таймеров для тактирования импульсов
            шагового двигателя
    \item ОЗУ не менее 8 кБ
    \item FLASH--память не менее 32 кБ
\end{itemize}

В результате был выбран контролер TMS320--F28035 со следующими параметрами
смотри таблицу (\ref{mcu_params}).
\begin{table}[ht!]
    \centering
    \begin{tabular}{|l|c|}
        \hline
        Характеристика & значение \\
        \hline \hline
        Процессор & 32--Bit CPU TMS320C28x \\
        Архитектура & Гарвардская \\
        Размер машинного слова, бит & 16 \\
        Endianness & Little Endian \\
        Языки програмирования & ASM, C, C++ \\
        Частота работы, МГц & 60 \\
        Время выполнения 1 инструкции, нс & 16.67 \\
        \hline
        FLASH, кБ & 64 \\
        SRAM,  кБ & 10 \\
        \hline
        Последовательные интерфейсы: & \\
        SCI  & есть \\
        SPI  & есть \\
        I2C  & есть \\
        LIN  & есть \\
        eCAN & есть \\
        \hline
        Модуль ШИМ (ePWM), шт & 14 \\
        \hline
        Сторожевой таймер & есть \\
        \hline
        АЦП & 12 разрядов \\
        \hline
        Встроенный датчик температуры & есть \\
        \hline
        Периферийные прерывания, шт & 3 \\
        \hline
        GPIO, шт  & 45 \\
        \hline
        Таймеры & 3 32--разрядных \\
        \hline
        Корпус & 80--ти выводной QFP \\
        \hline
        Напряжение питания, В & 3.3 \\
        \hline
    \end{tabular}
    \caption{Параметры TMS320--F28035}
    \label{mcu_params}
\end{table}


\paragraph{Усилитель мощности}
Усилитель мощности служит для усиления по мощности сигнала управления
двигателем. Он имеет гальваническую развязку всех управляющих сигналов с силовой
частью, что следует учесть при разработке устройства сопряжения с усилителем
мощности.
Был выбран импульсный усилитель мощности со следующими параметрами
смотри таблицу (\ref{drv_params}).

\begin{table}[ht!]
    \centering
    \begin{tabular}{|l|c|}
        \hline
        Характеристика & значение \\
        \hline \hline
        Напряжение питания, В & \\
        мин.  & 10.8 \\
        ном.  & 12   \\
        макс. & 13.2 \\
        \hline
        Силовое напряжение, В & \\
        мин.  & 10.8 \\
        ном.  & 12   \\
        макс. & 13.2 \\
        \hline
        Максимальная частота ШИМ, кГц & 500 \\
        \hline
        Максимальный рабочий ток на канал, А & 7 \\
        \hline
        Максимальный пусковой ток на канал, А & 15 \\
        \hline
        Температурный режим, \textcelsius & -40..+85 \\
        \hline
    \end{tabular}
    \caption{Параметры DRV8412}
    \label{drv_params}
\end{table}

\paragraph{Cвязь с бортовой шиной данных}
Согласно тезническому заданию плата должна управляться со стороны бортовой
\textit{ЭВМ} посредством интерфейса \textit{RS-485}, в полудуплексном режиме.
В кабеле используется экранированная витая пара.
Baud rate 115200. Выбран интерфейс \textit{RS-422/485 SN75176B} от
Texas Instruments.
В качетсве протокола передачи данных использовали модифицированный STX-ETX
протокол обмена. Состав сообщения преставлен в таблице (\ref{stx_etx_protocol}).

\begin{table}[ht!]
    \centering
    \begin{tabular}{|c|c|l|}
       \hline
       Тип & Длина, байт & Значение \\ \hline \hline
       STX    & 1      & \texttt{0x02} \\ \hline
       Длина  & 1      & \texttt{0x04..0xFF} \\ \hline
       Адрec  & 1      & \texttt{0x00..0x7F} \\ \hline
       Строка & 0..251 & \texttt{0x00..0x7F} \\ \hline
       ETX    & 1      & \texttt{0x03} \\ \hline
    \end{tabular}
    \caption{Протокол обмена сообщениями с бортовой сетью}
    \label{stx_etx_protocol}
\end{table}

\paragraph{Датчик обратной связи}
Выбранный энкодер \textit{ЛИР-137А 6250-05-ПИ} имеет TTL логику, а
микроконтролер имеет логику 0..+3.3. Для преобразования уровней поставили
6--разрядный двунаправленный преобразователь нарпяжения,
TXB0106PW Texas Instruments.

\paragraph{Датчик тока и датчик напряжения}
Датчик тока на плате управления выполнен на измерительном резисторе,
последовательно включенном в цепь каждой фазы. Так как управление происходит
сигналом ШИМ, то имеем 4 датчика тока в кажом канале (A, B, C, D).
Сигнал с сериесного резистора, проходящий через RC-фильтр и усилили на
операционном усилителе OPA2350 в инвертирующем подключении.

\paragraph{Шины питания}
Все микросхемы можно разделить на три группы по уровню питающего напряжения:
\begin{itemize}
    \item \textbf{+12В}:
        \begin{itemize}
            \item Драйвер ШИМ.
        \end{itemize}
        Исполльзовали линейный регулятор напряжения,
        \textit{TPS54160DGQ Texas Instruments}
    \item \textbf{+5В}:
        \begin{itemize}
            \item Энкодер \textit{ЛИР-137А 6250-05-ПИ}
            \item Операционный усилитель OPA2350
            \item Преобразователь сигнала энекодера \textit{TXB0106PW}
            \item Интерфейс \textit{RS-422/485 SN75176B}
        \end{itemize}
        Исполльзовали линейный регулятор напряжения
        \textit{UA78M05CDCY Texas Instruments}
    \item \textbf{+3.3В}:
        \begin{itemize}
            \item Преобразователь сигнала энекодера \textit{TXB0106PW}
            \item Микроконтролер \textit{TMS320F28035}
        \end{itemize}
        Исполльзовали линейный регулятор напряжения,
        \textit{UA78M33CDCY Texas Instruments}
\end{itemize}

\paragraph{Индикация состояния}
В числе прочего для целей отладки, тестирования и последущих процедур контроля
необхрдимо иметь визуальное средство для котроля состояния платы.
Перечень ``жизненно важных'' систем платы включает:
шину бортового питания,
шину питаня \textit{12В},
шину питаня \textit{5В},
шину питаня \textit{3.3В},
бортовая шина передачи информации \textit{RS-485},
микроконтролер.

Для этого на плату установлены:
\begin{enumerate}
    \item Индикатор статуса основной программы микроконтролера,
            светодиод зеленый \textit{KP-1608MGC}.
    \item Индикатор статуса бртовой сети связи,
            светодиод голубой \textit{KPH-1608PBC-A}.
    \item Индикатор критических ошибок микроконтролера,
            светодиод красный \textit{KPH-1608SEC}.
    \item Индикатор напряжения на шине бортового питания,
            светодиод белый \textit{KPT-1608QWF-E}.
    \item Индикатор напряжения на шине питания \textit{+12В},
            светодиод желтый \textit{KPT-1608YD}.
    \item Индикатор напряжения на шине питания \textit{+5В},
            светодиод зеленый \textit{KP-1608VGC(A)}.
    \item Индикатор напряжения на шине питания \textit{+3.3В},
            светодиод оранжевый \textit{KP-1608VS}.
\end{enumerate}

\subsubsection{Принципиальная схема платы упрывления}
\subsubsection{Конструкция платы управления}
\paragraph{Выбор материала заготовки}
В качестве диэлектрика выбрали стеклотекстолит
\textit{СТЕФ-У} ГОСТ 12652--74 \cite{GOST_12652_74}.
Длительная рабочая температура от \textit{-65\textcelsius}
до \textit{+155\textcelsius}.
Предназначен для работы в условиях нормальной относительной влажности
окружающей среды при напряжении свыше \textit{1000В}.

\paragraph{Выбор класса точности платы}
Для платы выбран класc точности 3.
Печатные платы 3--гo класса --- наиболее распространенные, поскольку, с одной
стороны, обеспечивают достаточно высокую плотность трассировки и монтажа, а с
другой --- для их производства требуется рядовое, хотя и специализированное,
оборудование.

\paragraph{Расстояние между элементами проводящего рисунка}
Согласно \cite[Табл. 7]{GOST_23751_86} для 3 класса точности,
фольгированного стеклотекстолита:

\begin{tabular}{|c|l|}
    \hline
    Мин. расстояние, мм & Проводники \\
    \hline
    0.4 & Силовые линии --- шины питания \textit{VCC} \\
    \hline
    0.2 & Остальные линии, включая внутренние шины питания \\
    \hline
\end{tabular}

\paragraph{Толщина печатного проводника}
Согласно \cite[Табл. 9]{GOST_23751_86} для 3 класса точности,
фольгированного стеклотекстолита и проводников с покрытием:

\begin{tabular}{|c|l|}
    \hline
    Мин. ширина, мм & Проводники \\
    \hline
    0.4 & Силовые линии --- шины питания \textit{VCC} \\
    \hline
    0.2 & Остальные линии, включая внутренние шиниы питания \\
    \hline
\end{tabular}

\paragraph{Размеры переходных металлизированых отверстий}
Все переходные метализированые отверстия на плате, для не монтажных целей:

\begin{itemize}
    \item Диаметр площадки на фольге под отверстия --- \textit{1.2мм.}
    \item Диаметр металлизированного отверстия --- \textit{0.7мм.}
    \item Диаметр сверления --- \textit{0.8мм.}
\end{itemize}
